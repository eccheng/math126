\documentclass[]{article}
\usepackage{lmodern}
\usepackage{amssymb,amsmath}
\usepackage{ifxetex,ifluatex}
\usepackage{fixltx2e} % provides \textsubscript
\ifnum 0\ifxetex 1\fi\ifluatex 1\fi=0 % if pdftex
  \usepackage[T1]{fontenc}
  \usepackage[utf8]{inputenc}
\else % if luatex or xelatex
  \ifxetex
    \usepackage{mathspec}
    \usepackage{xltxtra,xunicode}
  \else
    \usepackage{fontspec}
  \fi
  \defaultfontfeatures{Mapping=tex-text,Scale=MatchLowercase}
  \newcommand{\euro}{€}
\fi
% use upquote if available, for straight quotes in verbatim environments
\IfFileExists{upquote.sty}{\usepackage{upquote}}{}
% use microtype if available
\IfFileExists{microtype.sty}{\usepackage{microtype}}{}
\usepackage{graphicx}
\makeatletter
\def\maxwidth{\ifdim\Gin@nat@width>\linewidth\linewidth\else\Gin@nat@width\fi}
\def\maxheight{\ifdim\Gin@nat@height>\textheight\textheight\else\Gin@nat@height\fi}
\makeatother
% Scale images if necessary, so that they will not overflow the page
% margins by default, and it is still possible to overwrite the defaults
% using explicit options in \includegraphics[width, height, ...]{}
\setkeys{Gin}{width=\maxwidth,height=\maxheight,keepaspectratio}
\ifxetex
  \usepackage[setpagesize=false, % page size defined by xetex
              unicode=false, % unicode breaks when used with xetex
              xetex]{hyperref}
\else
  \usepackage[unicode=true]{hyperref}
\fi
\hypersetup{breaklinks=true,
            bookmarks=true,
            pdfauthor={},
            pdftitle={},
            colorlinks=true,
            citecolor=blue,
            urlcolor=blue,
            linkcolor=magenta,
            pdfborder={0 0 0}}
\urlstyle{same}  % don't use monospace font for urls
\setlength{\parindent}{0pt}
\setlength{\parskip}{6pt plus 2pt minus 1pt}
\setlength{\emergencystretch}{3em}  % prevent overfull lines
\setcounter{secnumdepth}{0}


\begin{document}

\section{Curvature}\label{curvature}

\subsection{Unit tangent vectors}\label{unit-tangent-vectors}

Something remarkable happens when parametrizing curves by arc length:
every tangent vector has length $1$.

We can do this more generally whenever we know that
$\mathbf f'(t)\neq 0$.

The unit tangent vector to the curve $\mathbf f(t)$ at a point $t=a$ is
\[\mathbf T(a)=\frac{\mathbf f'(a)}{|\mathbf
f'(a)|}.\]

Example: for the joey from the ``Distance'' lecture segment with
$\mathbf f'(t)=\langle 1,t,\sin(t)\rangle$, the unit tangent vector at
time $a$ is \[\mathbf
T(a)=\frac{1}{\sqrt{1+t^2+\sin^2(t)}}\langle 1,t,\sin(t)
\rangle.\]

\subsection{Practice}\label{practice}

Compute the unit tangent to the parabola $y=x^2$ at the point $(a,a^2)$.

Hint: one way to do this is to parametrize the path first, say using
$x=t$. Then calculate $\mathbf f'(t)/|\mathbf f'(t)|$. What if you
parametrize (half of) the parabola as $\langle\sqrt t,t\rangle$ instead?

\subsection{The benefits of all of
this}\label{the-benefits-of-all-of-this}

Why work with arc length and unit tangents?

\begin{itemize}
\itemsep1pt\parskip0pt\parsep0pt
\item
  Capture intrinsic geometric information, not artifacts of the choices
  we made in our description.
\item
  Recovers delicate static information about the shape.
\end{itemize}

For example: $\mathbf T'(t)\cdot\mathbf T(t)=0.$ I.e., $\mathbf T'(t)$
is perpendicular to $\mathbf T(t)$.

Indeed, \[0=\frac{d}{dt}1=\frac{d}{dt}|\mathbf
T(t)|=\frac{d}{dt}\left(\mathbf T(t)\cdot\mathbf
T(t)\right)=2\mathbf T(t)\cdot\mathbf T'(t).\]

Meaning: $\mathbf T'(t)$, unlike the acceleration in general, is always
changing the tangent to the curve in the ``most efficient'' way.

\subsection{Curvature}\label{curvature-1}

The curvature of the smooth parametric curve $\mathbf f(t)$ is defined
to be \[\kappa(t)=\left|\frac{d\mathbf
T(t)}{ds}\right|,\] where $s$ is the arc length function.

Since $s'(t)=|\mathbf f'(t)|$, we also have
$\kappa(t)=|\mathbf T'(t)|/|\mathbf f'(t)|$.

A lot to digest!

When $\mathbf f$ is already paramterized by arc length, this simplifies
to \[\kappa(s)=|\mathbf f''(s)|,\] the acceleration.

In practice, this is \emph{not} how you will compute it (because you
won't have paths parametrized by arc length most of the time).

\subsection{Examples}\label{examples}

\subsubsection{\emph{The Circle of Radius $r$} and \emph{The Parabola
Fights
Back}.}\label{the-circle-of-radius-r-and-the-parabola-fights-back.}

The equations parametrizing with arc length:
$\mathbf f(s)=\langle r\cos(s/r),r\sin(s/r)\rangle$.

Thus, \[\kappa(s)=\left|\left\langle
-\frac{1}{r}\cos\left(\frac{s}{r}\right),
-\frac{1}{r}\sin\left(\frac{s}{r}\right)\right\rangle\right|=\frac{1}{r}.\]

Makes sense: the curvature of a circle of large radius is small. After
all, the path is basically a straight line!

What about for something like the parabola? Try it. You might consider
using $\mathbf f(t)=\langle t,t^2\rangle$ and
$\kappa=|\mathbf T'|/|\mathbf f'|$, together with your calculation of
$\mathbf T$ from before. What a mess!

\subsection{Curvature in practice}\label{curvature-in-practice}

Mathematicians have thought about this one pretty hard, and here is what
turns out to happen:

The curvature of the smooth path parametrized by $\mathbf f(t)$ is
\[\kappa(t)=\frac{|\mathbf f'(t)\times\mathbf f''(t)|}{|\mathbf
f'(t)|^3}.\] Any book or the internet contains a proof!

We can dispatch the parabola $\mathbf f(t)=\langle t,t^2\rangle$:

\[\kappa(t)=\frac{|\langle 1,2t,0\rangle\times\langle
0,2,0\rangle|}{(1+4t^2)^\frac{3}{2}}=\frac{2}{(1+4t^2)^\frac{3}{2}}.\]

\subsection{Do one!}\label{do-one}

Recall the motion of the joey:
$\mathbf f(t)=\langle t,\frac{1}{2}t^2,2-\cos(t)\rangle$.

Calculate the limit of the curvature as $t\to\infty$.

Formulas for curvature: \[\kappa(t)=\left|\frac{d\mathbf
T}{ds}\right|=\frac{|\mathbf T'(t)|}{|s'(t)|}=\frac{|\mathbf
f'(t)\times\mathbf f''(t)|}{|\mathbf f'(t)|^3}\]


\end{document}
